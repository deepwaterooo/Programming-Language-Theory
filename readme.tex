% Created 2016-05-19 Thu 01:55
\documentclass[9pt,b5paper]{article}
\usepackage{fontspec}
\setmainfont{STSong}
\usepackage{graphicx}
\usepackage{xcolor}
\usepackage{xeCJK}
\setCJKmainfont{STSong}
\usepackage{longtable}
\usepackage{float}
\usepackage{textcomp}
\usepackage{geometry}
\geometry{left=0cm,right=0cm,top=0cm,bottom=0cm}
\usepackage{multirow}
\usepackage{multicol}
\usepackage{listings}
\usepackage{algorithm}
\usepackage{algorithmic}
\usepackage{latexsym}
\usepackage{natbib}
\usepackage[xetex,colorlinks=true,CJKbookmarks=true,linkcolor=blue,urlcolor=blue,menucolor=blue]{hyperref}


\lstset{language=c++,numbers=left,numberstyle=\tiny,basicstyle=\ttfamily\small,tabsize=4,frame=none,escapeinside=``,extendedchars=false,keywordstyle=\color{blue!70},commentstyle=\color{red!55!green!55!blue!55!},rulesepcolor=\color{red!20!green!20!blue!20!}}
\author{deepwaterooo}
\date{\today}
\title{Programming Language Theory -- Summer 2016}
\hypersetup{
  pdfkeywords={},
  pdfsubject={},
  pdfcreator={Emacs 24.5.1 (Org mode 8.2.7c)}}
\begin{document}

\maketitle
\tableofcontents


\section{Introduction}
\label{sec-1}
\begin{itemize}
\item A reposition for tracking summer2016 Programming Language Theory course.
\item hw1: DrRacket Image/Rsound Animation.
\end{itemize}

\section{References}
\label{sec-2}
\begin{itemize}
\item framework \url{https://github.com/NetEase/lively-logic}
\item \url{https://www.youtube.com/watch?v=SCh0zmP6R5A}
\item \url{https://www.youtube.com/watch?v=ayqhX9UA6FY}
\item \url{http://racket.tchen.me/practical-racket.html}
\item 
\item 
\item 
\item 
\item 
\item 
\end{itemize}
% Emacs 24.5.1 (Org mode 8.2.7c)
\end{document}